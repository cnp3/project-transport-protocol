
\documentclass[a4paper,12pt]{article}
\usepackage[margin=2cm]{geometry} %
\usepackage[french]{babel} %
\usepackage[T1]{fontenc} %
\usepackage[utf8]{inputenc} %
\usepackage{fancyhdr} %
\usepackage{hyperref} %
\usepackage{paralist} %
\usepackage{url}

\usepackage{pifont,mdframed}

\newenvironment{warning}
{\par\begin{mdframed}[linewidth=2pt,linecolor=red]%
		\begin{list}{}{\leftmargin=1cm
				\labelwidth=\leftmargin}\item[\Large\ding{43}]}
		{\end{list}\end{mdframed}\par}


\def\mytitle{TRTP: Truncated Reliable Transport Protocol}\title{\mytitle}
\def\mydate{Sept. 2018}\date{\mydate}
\def\myauthor{LINGI1341}\author{\myauthor}

\lfoot{} %
\cfoot{} %
\rfoot{\thepage} %
\lhead{%
    \large
    \href{https://moodleucl.uclouvain.be/course/view.php?id=7995
         }{Projet 1\space\myauthor\space-- Réseaux informatiques}
}
\chead{} %
\rhead{%
    \large\mydate%
}

\newcommand{\must}{\textbf{DOIT}}
\newcommand{\should}{\textbf{DEVRAIT}}

\newcommand{\pdata}{\texttt{PTYPE\_DATA}}
\newcommand{\pack}{\texttt{PTYPE\_ACK}}
\newcommand{\pnack}{\texttt{PTYPE\_NACK}}

\newcommand{\sender}{\texttt{sender}}
\newcommand{\receiver}{\texttt{receiver}}

\begin{document}
\rmfamily %
\pagestyle{fancy} %

\begin{center}
    \huge\bfseries\mytitle%
    \vspace{5ex}
\end{center} %

\section{Description du projet}
Une célèbre entreprise de céréales, Crousti-Croc, a récemment installé un nouveau data-center dans la région.
Celui-ci permettra à la compagnie de prévoir le rendement de ses cultures grâce à des simulations.
Les calculs sont complexes, et ceux-ci requièrent de la coordination entre les différentes machines.
La communication entre les serveurs requiert donc un protocole de transport.

L'entreprise veut maximiser le temps CPU utilisé pour les simulations.
Pour se faire, elle a conçu Truncated Reliable Transport Protocol (TRTP), un protocole de transport \textbf{fiable} basé sur des segments \textbf{UDP}.
Elle désire une implémentation performante dans le langage \textbf{C} qui ne contient \textbf{aucune fuite de mémoire}\footnote{Celles-ci peuvent faire crasher des machines de calculs, impliquant un coût de maintenance et de retard de simulation conséquent.}.
TRTP permettra de réaliser des transferts fiables de fichiers, utilisera la
stratégie du \textbf{selective repeat}\footnote{Le selective repeat implique
uniquement que le \receiver\ accepte les paquets hors-séquence et les stocke
dans un buffer s'ils sont dans la fenêtre de réception. Par contre, ce
protocole-ci ne permet pas au \receiver\ d'indiquer au \sender\ quels sont les
paquets hors-séquence qu'il a reçu.} et permettra la \textbf{troncation de payload}\footnote{Tronquer les payloads peut permettre au réseau de décongestionner ses buffers tout en fournissant un mécanisme de feedback rapide. Cette idée est plus détaillée dans un article récent~\cite{handley2017re}.}.
Comme l'architecture du réseau de l'entreprise est \textbf{uniquement basée sur IPv6}, TRTP devra fonctionner au dessus de ce protocole.

Ce projet est à faire par \textbf{groupe de deux}.
Deux programmes sont à réaliser, permettant de faire un transfert de données entre
deux machines distantes.

\section{Spécifications}
\subsection{Protocole}
Le format des segments du protocole est visible sur la figure~\ref{fig:format}. Ils 
se composent des champs dans l'ordre suivant:
\begin{description}
    \item[Type] Ce champ est encodé sur 2 bits. Il indique le type du paquet,
        trois types sont possibles:
        \begin{enumerate}[(i)]
            \item \pdata\ = 1, indique un paquet contenant des données;
            \item \pack\ = 2, indique un paquet d'acquittement de données
                reçues.
            \item \pnack\ = 3, indique un paquet de non acquittement d'un packet de données tronquées (càd, avec le champ TR à 1).
        \end{enumerate}
        Un paquet avec un autre type \must\ être ignoré;
    \item[TR] Ce champ est encodé sur 1 bit, et indique si le réseau a tronqué un payload initialement existant dans un packet \pdata\ .  
	    La réception d'un paquet avec ce champ à 1 provoque l'envoi d'un paquet \pnack\ .
	    Un paquet d'un type autre que \pdata\ avec ce champ différent de 0 \must\ être ignoré;
    \item[Window] Ce champ est encodé sur 5 bits, et varie donc dans l'intervalle [0, 31].
        Il indique la taille de la fenêtre de réception de l'émetteur de ce paquet.
        Cette valeur indique le nombre de places vides dans le buffer de réception de
        l'émetteur du segment, et peut varier au cours du temps.
        Si un émetteur n'a pas de buffer de réception, cette valeur \must\ être mise à 0.
        Un émetteur ne peut envoyer de nouvelles données avant d'avoir reçu un acquit pour le
        paquet précédant que si le champs Window du dernier paquet provenant
        du destinataire était non nul. Lors de la création d'un nouvelle connexion,
        l'émetteur initiant la connexion \must\ considérer que le destinataire avait annoncé une
        valeur initiale de Window valant 1.
    \item[Seqnum] Ce champ est encodé sur 8 bits, et sa valeur varie dans l'intervalle [0, 255].
        Sa signification dépend du type du paquet.
        \begin{description}
            \item[\pdata] Il correspond au numéro de séquence de ce paquet de données.
                Le premier segment d'une connexion a le numéro de séquence 0. Si le numéro
                de séquence ne rentre pas dans la fenêtre des numéros de séquence autorisés
                par le destinataire, celui-ci \must\ ignorer le paquet;
            \item[\pack] Il correspond au numéro de séquence du prochain numéro de séquence
                attendu (c-à-d (le dernier numéro de séquence + 1) $\% 2^8$).
                Il est donc possible d'envoyer un seul paquet \pack\ qui sert d'acquittement
                pour plusieurs paquets \pdata\ (principe des acquis
                cumulatifs);
            \item[\pnack] Il correspond au numéro de séquence du paquet tronqué qui a été reçu. Si il ne rentre pas dans la fenêtre des numéros de séquence envoyés par l'émetteur, celui-ci \must\ ignorer le paquet.
        \end{description}
        Lorsque l'émetteur atteint le numéro de séquence 255, il recommence à 0;
    \item[Length] Ce champ est encodé sur 16 bits en network-byte order, et sa
        valeur varie dans l'intervalle [0, 512].
        Il dénote le nombre d'octets de données dans le payload.
        Un paquet \pdata\ avec ce champ à 0 et dont le numéro de séquence correspond
        au dernier numéro d'acquittement envoyé par le destinataire signifie que le transfert est
        fini. Une éventuelle troncation du paquet ne change pas la valeur de ce champ. Si ce champ 
        vaut plus que 512, le paquet \must\ être ignoré;
    \item[Timestamp] Ce champs est encodé sur 4 octets, et représente une
        valeur opaque donc sans endianness particulière. Pour chaque paquet
        \pdata\, l'émetteur d'un paquet choisit une valeur à mettre dans ce
        champs. Lorsque le destinataire envoie un paquet \pack\, il indique
        dans ce champs la valeur du champs Timestamp du \textbf{dernier} paquet
        \pdata\ reçu. La signification de la valeur est laissée libre aux
        implémenteurs;
    \item[CRC1] Ce champs est encodé sur 4 octets, en network byte-order.
    Ce champ contient le résultat de l'application de la fonction
    CRC32\footnote{Le diviseur (polynomial) de cette fonction est
    	$x^{32} + x^{26} + x^{23} + x^{22} + x^{16} + x^{12} + x^{11} + x^{10} + x^8 + x^7 + x^5 + x^4 + x^2 + x + 1$, la représentation
    	normale de ce polynôme (en network byte-order) est $0\times04C11DB7$. L'implémentation
    	la plus courante de cette fonction se trouve dans \texttt{zlib.h}, mais de nombreuses
    	autres existent.}
    au header avec le champ TR mis à 0\footnote{En règle générale, un routeur commencera à tronquer les paquets lorsque ses charges réseau et/ou CPU deviendront importantes. On voudrait donc éviter que ce dernier ait à recalculer le CRC (et donc perdre du temps CPU) si le champ TR change de valeur.}, juste avant qu'il ne soit envoyé sur le réseau.
    À la réception d'un paquet, cette fonction doit être recalculée sur le header avec le champ TR mis à 0, et 
    le paquet \must\ être ignoré si les deux valeurs diffèrent.
    \item[Payload] Ce champ contient au maximum 512 octets.
        Il contient les données transportées par le protocole. Si le champ TR est 0, sa taille est
        donnée par le champs Length; sinon, sa taille est nulle. 
    \item[CRC2] Ce champs est encodé sur 4 octets, en network byte-order.
        Ce champ contient le résultat de l'application de la fonction
        CRC32 à l'éventuel payload, juste avant qu'il ne soit envoyé sur le réseau.
        Ce champ n'est présent que si le paquet contient un payload et qu'il n'a pas été tronqué (champ TR à 0).
        À la réception d'un paquet avec ce champ, cette fonction doit être recalculée sur le payload, et 
        le paquet \must\ être ignoré si les deux valeurs diffèrent.
\end{description}
\begin{figure}
    \centering
    \begin{minipage}{50ex}
\begin{verbatim}
 01 2 3 7 8     15 16     23 24    31
+--+-+---+--------+---------+--------+
|Ty|T|Win|  Seq   |      Length      |
|pe|R|dow|  num   |                  |
+--+-+---+--------+---------+--------+
|        Timestamp (4 octets)        |
|                                    |
+------------------------------------+
|           CRC1 (4 octets)          |
|                                    |
+------------------------------------+
|                                    |
|                                    |
+                                    +
.       Payload (max 512 octets)     .
.                                    .
.                                    .
+------------------------------------+
|     CRC2 (4 octets, optionnel)     |
|                                    |
+------------------------------------+
\end{verbatim}
    \end{minipage}
    \caption{Format des paquets du protocole}\label{fig:format}
\end{figure}

\medskip
Les communications peuvent relier deux machines qui se trouvent soit au sein d'un même data-center, soit dans différents bâtiments répartis à travers le monde.
L'entreprise veut donc que TRTP fonctionne correctement dans le modèle de réseau suivant:
\begin{enumerate}
    \item Un segment de données envoyé par un hôte est reçu \textbf{au plus une fois}
        (pertes mais pas de duplication);
    \item Le réseau peut \textbf{corrompre} les segments de données de façon aléatoire;
    \item Le réseau peut \textbf{tronquer} le payload des paquets de façon aléatoire;
    \item Soit deux paquets, $P_1$ et $P_2$, si $P_1$ est envoyé avant $P_2$, il n'y
        a \textbf{pas de garantie} concernant l'\textbf{ordre} dans lequel ces paquets seront reçus
        à la destination;
    \item La \textbf{latence} du réseau pour acheminer un paquet varie dans
    l'intervalle \textbf{[0,2000]} (ms).
\end{enumerate}

\subsection{Programmes}
L'implémentation du protocole devra permettre de réaliser un transfert de données
unidirectionnel au moyen de deux programmes, \sender\ et \receiver.
Ces deux programmes devront être produit au moyen de la cible par défaut d'un Makefile.
Cette implémentation \must\ fonctionner sur les ordinateurs \textbf{Linux} de la salle Intel, bâtiment Réaumur\footnote{Les machines peuvent être accédées via SSH; consultez \url{https://wiki.student.info.ucl.ac.be/Mat\%c3\%a9riel/SalleIntel}}.

Les deux programmes nécessitent au minimum deux arguments pour se lancer:\\
\sender\ \texttt{hostname port}, avec \texttt{hostname} étant le nom de domaine ou l'adresse
IPv6 du \receiver, et \texttt{port} le numéro de port UDP sur lequel le
\receiver\ a été lancé.\\
\receiver\ \texttt{hostname port}, avec \texttt{hostname} le nom de domaine où l'adresse
IPv6 sur laquelle le \receiver\ doit accepter une connexion (\texttt{::} pour écouter sur toutes les
interfaces), et \texttt{port} le port UDP sur lequel le \receiver\ écoute.\\

Enfin, ces deux programmes supporteront l'argument optionnel suivant:\\
\texttt{-f X}, afin de spécifier un fichier X, \textbf{potentiellement binaire}, à envoyer par le \sender\
ou le fichier dans lequel sauver les données reçues par le \receiver.
Si cet argument n'est pas spécifié, le \sender\ envoie ce qu'il lit sur l'entrée
standard (CRTL-D pour signaler EOF), et le \receiver\ affiche ce qu'il reçoit sur
sa sortie standard.

Les deux programmes doivent utiliser la \textbf{sortie d'erreur standard} s'ils veulent afficher
des informations à l'utilisateur.

\medskip
Voici quelques exemples pour lancer les programmes:\\
\begin{tabular}{lp{.5\textwidth}l}
    \sender\ \texttt{::1 12345 <{} fichier.dat} & Redirige le contenu du fichier fichier.dat
                                       sur l'entrée standard, et l'envoie sur le
                                       \receiver\ présent sur la même machine
                                       (\texttt{::1}),
                                       qui écoute sur le port 12345 \\
    \sender\ \texttt{-f fichier.dat ::1 12345} & Idem, mais sans passer par l'entrée standard\\
    \sender\ \texttt{-f fichier.dat localhost 12345} &
    Idem, en utilisant un nom de domaine (localhost est défini dans
    /etc/hosts)\\
    \receiver\ \texttt{:: 12345 > fichier.dat 2> log.log} & Lance un receiver qui écoute sur toutes
                                       les interfaces, et affiche le contenu du transfert sur
                                       la sortie standard, qui est redirigée dans fichier.dat, alors
                                       que les messages d'erreur et de log sont redirigés dans
                                       log.log\\
\end{tabular}

\section{Tests}
\subsection{INGInious}
Des tests INGInious seront fournis pour vous aider à tester deux facettes du
projet:
\begin{enumerate}
    \item La création, l'encodage et le décodage de segments;
    \item L'envoi et la réception de donnée sur le réseaux, multiplexés sur un
        socket.
\end{enumerate}

\subsection{Tests individuels}
Les tests INGInious ne seront pas suffisant pour tester votre implémentation. Il vous est donc
demandé de tester par vous-même votre code, afin de réaliser une suite de test, et de le documenter
dans votre rapport.

\subsection{Tests d'interopérabilité}
Vos programmes doivent être inter-opérables avec les implémentations d'autres groupes.
Vous ne pouvez donc pas créer de nouveau type de segments, ou rajouter des méta-données
dans le payload. 
Deux possibilités de tester l'interopérabilité de votre implémentation s'offrent à vous.
\begin{enumerate}
	\item Durant une partie de la semaine précédent la première soumission, vous aurez la possibilité de tester votre \sender\ avec le \receiver\ de référence.
	\item Une semaine avant la remise de la deuxième soumission, vous \textbf{devrez} tester votre implémentation
	avec 2 autres groupes (votre \sender\ et leur \receiver, votre \receiver\ et leur \sender).
\end{enumerate}

\section{Planning et livrables}

\medskip
\textbf{Première soumission, 22/10 à 8h} %

\begin{enumerate}
    \item Rapport (max 4 pages, en PDF), décrivant l'architecture générale de votre
        programme, et répondant \textbf{au minimum} aux questions suivantes:\footnote{Le rapport sera lu par des experts mandatés par l'entreprise qui connaissent le sujet. Soyez donc \textbf{objectifs} et évitez de réintroduire le contexte du projet dans votre rapport. Toutefois, libre à vous mettre en avant une partie du projet non listée dans la liste ci-dessous que vous trouvez pertinente.}
        \begin{itemize}
            \item Que mettez-vous dans le champs Timestamp, et quelle
                utilisation en faites-vous?
            \item Comment réagissez-vous à la réception de paquets \pnack\ ?
            \item Comment avez-vous choisi la valeur du retransmission timeout?
            \item Quelle est la partie critique de votre implémentation, affectant
                la vitesse de transfert?
            \item Quelles sont les performances de votre protocole? (\textbf{évaluation} graphique et/ou chiffrée, scénarios explorés, explication de la méthodologie)
            \item Quelle(s) stratégie(s) de tests avez-vous utilisée(s)?
        \end{itemize}
    \item Implémentation des deux programmes permettant un transfert de données en
        utilisant le protocole.	
    \item Suite de tests des programmes.
    \item Makefile dont la \textbf{cible par défaut} produit les deux programmes
        dans le répertoire courant avec comme noms \sender\ et \receiver, et
        dont la cible \textbf{tests} lance votre suite de tests.
\end{enumerate}

\medskip
\textbf{Tests d'inter-opérabilité, durant les séances de TP de la semaine du 22/10 au 26/10, salle Intel (planning à confirmer)}
\medskip

\medskip
\textbf{Deuxième soumission, 2/11/2016}

Même critères que pour la première condition, le rapport devant décrire en plus
le résultat des tests d'interopérabilité en annexe, ainsi que les changements
effectués au code si applicable.

\medskip
\textbf{Format des livrables}

La soumission fera en une seule archive \textbf{ZIP}, respectant le format
suivant:
\begin{center}
\begin{tabular}{lp{.5\textwidth}l}
    \texttt{/} & Racine de l'archive \\
    \hskip 2ex\texttt{Makefile} & Le Makefile demandé \\
    \hskip 2ex\texttt{src/} & Le répertoire contenant le code source des deux programmes \\
    \hskip 2ex\texttt{tests/} & Le répertoire contenant la suite de tests \\
    \hskip 2ex\texttt{rapport.pdf} & Le rapport \\
    \hskip 2ex\texttt{gitlog.stat} & La sortie de la commande \texttt{git log -{}-stat} \\
\end{tabular}
\end{center}
Cette archive sera nommée \texttt{projet1\_nom1\_nom2.zip}, où \texttt{nom1/2} sont les
noms de famille des membres du groupe, et sera à mettre sur deux tâches dédiées sur Inginious (une pour chaque soumission) qui vérifieront le format de votre archive.
Si celle-ci ne passe pas les tests de format, \textbf{votre soumission ne sera pas considérée pour l'évaluation!}
Les liens de ces tâches vous seront communiqués en temps utile.

Il est demandé de réaliser le projet sous un gestionnaire de version (typiquement \texttt{git}).

\begin{warning}
	\textbf{Important:} l'évaluation tiendra compte de vos deux soumissions, et pénalisera les projets qui ont deux soumissions trop différentes.
	La première soumission n'est donc \textbf{PAS} facultative.
\end{warning}

\section{Evaluation}

La note du projet sera composée des trois parties suivantes:
\begin{enumerate}
    \item Implémentation: Vos programmes fonctionnent-ils correctement? Qualité de la suite de
        tests? Que se passe t-il quand le
        réseau est non-idéal? Sont-ils inter-opérables? \ldots

        Le respect des spécifications (arguments, fonctionnement du protocole,
        formats des segments, structure de l'archive, \ldots) est impératif
        à la réussite de cette partie!
    \item Rapport;
    \item Code review individuelle du codes de deux autres groupes. Effectuée durant
        la semaine suivant la remise du projet. Vous serez noté sur la pertinence
        de vos reviews. Plus d'informations suivront en temps voulu.
\end{enumerate}

\section{Ressources utiles}
Les manpages des fonctions suivantes sont un bon point de départ pour
implémenter le protocole, ainsi que pour trouver d'autres fonctions utiles:
\texttt{socket, bind, getaddrinfo, connect, send, sendto, sendmsg, recv,
    recvfrom, recvmsg, select, poll, getsockopt, shutdown, read, write, fcntl,
getnameinfo, htonl, ntohl, getopt}

\medskip

\cite{beej}~est un tutoriel disponible en ligne, présentant la plupart des appels
systèmes utilisées lorsque l'on programme en C des applications interagissant
avec le réseau.
\textbf{C'est probablement la ressource qui vous sera le plus utile pour commencer votre projet.}

\cite{stevens}~et~\cite{linux}~sont deux livres de références sur la
programmation système dans un environnement UNIX, disponibles en bibliothèque
INGI.\@

\cite{socket}~est le livre de référence sur les sockets TCP/IP en C,
disponible en bibliothèque INGI.\@

\cite{howtosend}~et~\cite{howtoreceive}~sont deux mini-tutoriels pour la création d'un serveur et
d'un client utilisant des sockets UDP.\@

\cite{select}~et \texttt{man select\_tut} donnent une introduction à l'appel système
\texttt{select}, utile pour lire et écrire des données de façon non-bloquante.

\bibliographystyle{plainurl}
\bibliography{biblio.bib}


\end{document}
